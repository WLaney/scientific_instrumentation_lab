% This is style of APS journal, physicists should aim to use it
\documentclass[prl,12pt,notitlepage,aps,onecolumn,superscriptaddress]{revtex4-1}

%---------------------------------------------------------------
\usepackage{listings} % allow to include nicely formated listings
\usepackage{xcolor}   % we can add and use colors
\usepackage{fullpage} % default article style is to greedy about margins
\usepackage{graphicx} % REQUIRED TO WORK WITH FIGURES
\usepackage{amsmath,amssymb} % much better math and equation handling
\usepackage{booktabs}
\usepackage{float}
%---------------------------------------------------------------

\begin{document}
%---------------------------------------------------------------
\title{Lab 04 Report}
\author{William Laney \& Ben Kincaid}
\date{September 28, 2016}
\maketitle

\section{Group Member Roles}

Circuit Architect: William Laney

Resource Manager (collected and tested componets): Ben Kincaid

Hardware Specialist (assembled the circuit): Ben Kincaid

Programer: William Laney

QA Specialist William Laney \& Ben Kincaid

\section{PWM testing}
We began this lab by connecting the osciscope to GPIO pin 18 in order to test the pulse width modficcation (PWM) funcunality of the Pi. We wrote a simple program that would turn on PWM for 5second with a duty cycle anf frquencies we speficed. We began by setting the frequency to 1Mhz, at this frequency any duty cycle above 10 appeared as a duty cycle of 100 on the scope. At a duty cycle of 10 on the scope it appeared to be closer to a duty cycle of 50. We determed from these test that 1Mhz is far outside the top frequncy rang of the Pi. We then repated this test at 100kHz and a duty cycle of 50. We found that the output had a frequncy of 6-8kHz. This is below the expected output freuqncy. We then lowerd the frequency again to about 7khz. and found that the output was approxamenlty what was expected but would sometime flicker and jump frequnciues unexpectedly. Contuinung test liks this we found that the PMW starts to become slightly unstable at 5kHz and caps out at about 10kHz.

Finaly we set the duty cycle of our PMW to 0 at a series of frequencies and found that out Pi would still outpus semi-regiular small pulses. This is not expected behavior, we expected that at 0 there would no pulses from the pi. We do not know the cause of this behavior and were not able to check on other pis if they show similar behavior. This did show us that out poi had a lower rang for it's duty cycle of 1. 

A picture of the PWM output we saw at 10Khz, 50\% duty cycle can be seen in figure 1.

\begin{figure}[h]
\begin{center}
\includegraphics[width=.4\columnwidth]{pwm.jpg}
\end{center}
\caption{\label{fig:pic} Pi PWM outpu}
\end{figure}

\section{DC motor}
With the PWM ablity of the Pi's profiled we stared to power a DC motor with the Pi and messure it's rotaion frequency versus the duty cycle. To do this we built the circuit seen in figure 2. The MOSFET was used to allow the Pi to power the motor without the PI being exposed to the full 7V used for the moter. 

\begin{figure}[h]
\begin{center}
\includegraphics[width=.4\columnwidth]{lab4-1.png}
\end{center}
\caption{\label{fig:pic}DC motor circuit}
\end{figure}

Two buttons were used to control the duty cycle outputed by the pi. The code for this can be seen in listing 1.

A photodiode and LED were used to messure the rotational frequncy of the mototr. This was done by putting a proper on the moter, an LED below the motoer and a photodiode above it, conected to the scope. When the propler corssed over the photodiode we would see a pip from the light of the LED on the scope. Using the time between these pips we can calculate the rotational velocity of the motor. An image of this pips can be seen in figure 3. We steped the duty cycle from 5-95\%. The results of this can be seen in table 1, and a graph in figure 5.  A picutre of the circuit can be seen in figure 4.

\begin{figure}[h]
\begin{center}
\includegraphics[width=.4\columnwidth]{photo.jpg}
\end{center}
\caption{\label{fig:pic} Photodiode output}
\end{figure}

\begin{figure}[h]
\begin{center}
\includegraphics[width=.4\columnwidth]{circuit_pic.jpg}
\end{center}
\caption{\label{fig:pic} Picture of DC motor circuit}
\end{figure}
 
We expected the data to show that as duty cycle increaed so would angluar velocity, and this is true until we got past a duty cycle of about 35. Then the angular velocity remained aproamnety constatn despite the changes in duty cycle. We belive this is because we reached the maximum angular velocity of the motor. Furthermore we belive we may have acutaly been burning the motor oout over the ocrse of the experemtn. During the experment we heard the motoer make unusal high pitch noises in addion to relaseing a platic smell. We belive this is because the gears in the motor were breaking off during th course of the experment. The gear brekaing would limit the maximum angular velocity of the motor. These results show that with our Pi and motor combination we can only reliably control the motor appromanelty 60 to 100 rad/sec. Our Pi has trouble mataining a duty cyle low enough to get the motoer below 60rad/sec, and the motor does not appear to be able to get over 100rad/sec.

% Table generated by Excel2LaTeX from sheet 'Sheet1'
\begin{table}[h]
  \centering
  \caption{PWM vs Angular Velocity}
    \begin{tabular}{c|c|c}
    \toprule
    duty cycle & time per revolution (ms) $\pm$5ms & Angular velocity (rad/sec) \\
    \midrule
    5     & 110   & 57.1199 \\
    15    & 100   & 62.8319 \\
    25    & 75    & 83.7758 \\
    35    & 65    & 96.6644 \\
    45    & 95    & 66.1388 \\
    55    & 85    & 73.9198 \\
    65    & 85    & 73.9198 \\
    75    & 90    & 69.8132 \\
    85    & 85    & 73.9198 \\
    95    & 90    & 69.8132 \\
    \bottomrule
    \end{tabular}%
  \label{tab:addlabel}%
\end{table}%

\begin{figure}[h]
\begin{center}
\includegraphics[width=.6\columnwidth]{graph.png}
\end{center}
\caption{\label{fig:pic} Angular velocity (rad/sec) vs. duty cycle }
\end{figure}

\section{Conclusion}
In this lab we learned about pulse width modfication, and speficaly the pulse width modfication capablities of the raspberry Pis. We learned how to find the limitations of the Pi's PWM, and what they are. We also learend how to use physical hardware buttons to interact with the Pis. Additonaly we learned how to use high voltage devices with a low volatge control through a MOSFET. Finally we learned how to desgine a test sysem for a DC motor.

\lstinputlisting[language=Python, caption=PMW-buttons, label=amb]{laney_desgine_assigmnet4_final.py}


\end{document}