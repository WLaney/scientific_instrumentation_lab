% This is style of APS journal, physicists should aim to use it
\documentclass[prl,12pt,notitlepage,aps,onecolumn,superscriptaddress]{revtex4-1}

%---------------------------------------------------------------
\usepackage{listings} % allow to include nicely formated listings
\usepackage{xcolor}   % we can add and use colors
\usepackage{fullpage} % default article style is to greedy about margins
\usepackage{graphicx} % REQUIRED TO WORK WITH FIGURES
\usepackage{amsmath,amssymb} % much better math and equation handling
\usepackage{booktabs}
\usepackage{float}
\usepackage{listings}
\lstset{language=Python} 
%---------------------------------------------------------------
%creat command that \ohm creats upcase omega (the ohm symbol)
\newcommand{\ohm}{$\Omega$ }

\begin{document}
%---------------------------------------------------------------
\title{Lab 03 Report}
\author{William Laney \& Ethan Copa-Vojtko}
\date{September 21, 2016}
\maketitle

\section{Group Member Roles}
Circuit Architect: William Laney

Resource Manager (collected and tested componets): Ethan Copa-Vojtko

Hardware Specialist (assembled the circuit): William Laney

QA Specialist: Ethan Copa-Vojtko

\section{Using a Frequency Mixers}
We began this lab by connecting two function generators to the multiplier circuit inputs. We set these function generators to have a frequency of about 50kHz, and a peak to peak voltage of 5V. An image of the output of this circuit can be seen in figure 1. 

\begin{figure}[h]
\begin{center}
\includegraphics[width=.4\columnwidth]{bfilter.jpg}
\end{center}
\caption{\label{fig:pic} Out put of mixer without filtering}
\end{figure}

This wave form conistes of two parts, a high and low frequency part. The low frequency part corosoponds to the diffrence between the two input frequeies of the sine waves. To isolate this frequecies we created a RC low pass filter. We expted the diffrence between the two freques to be around around 1kHz and the part we were trying to filter out to be near 200kHz. To isolate this signal we decided to build the circuit to have a cut off frequency of about 25kHz. to achive this we used a  150\ohm resistor and a 0.04$\mu$F capacitor. This gave us a cut off frequency of 26.5kHz. An image of the output of our cirtucit after the filter can be seen in figure 2.

\begin{figure}[h]
\begin{center}
\includegraphics[width=.4\columnwidth]{afilter.jpg}
\end{center}
\caption{\label{fig:pic} Out put of mixer with filtering}
\end{figure}

 We foudn the diffrence between our two frequencies to be about 500hz. This is constent with what we expected. We did not however get out a very periodical signal, we belive that htis is due to woble in one or both of our function generators. A Digrame and picture of the cirucit can be seen in figures 3 and 4.
 
\begin{figure}[h]
\begin{center}
\includegraphics[width=.4\columnwidth]{lab3-1.png}
\end{center}
\caption{\label{fig:pic} Frequncy mixing circuit digrame}
\end{figure} 
 
\begin{figure}[h]
\begin{center}
\includegraphics[width=.4\columnwidth]{circuit.jpg}
\end{center}
\caption{\label{fig:pic} Frequncy mixing circuit}
\end{figure}

\section{Raspberry Pi 3}
In this part of the lab we began familarisng oursleves using Raspberry Pi 3's. We began by connecting the pi to a mouse, keyboard, and mointer. We then booted up the pi and opened idle. TO confirm that the pie was conected correctly we created a circuit with a 1k\ohm reisotr and an LED from the pi's constant 3.3V out put to ground. The LED turned on showing that the pi was working correctly. We then connected the LED circuit to GPIO pin 18 and wrote a short oython program to blink the LED. The code we used can be seen in listing 1. A circuit digrame can be seen in figure 5.
\begin{figure}[h]
\begin{center}
\includegraphics[width=.4\columnwidth]{lab3-2.png}
\end{center}
\caption{\label{fig:pic} Single LED pi cirucit}
\end{figure} 

\begin{lstlisting}[caption=blinking LED, label=amb]

import RPi.GPIO as GPIO
import time
GPIO.setmode(GPIO.BCM)
GPIO.setup(18, GPIO.OUT)
n=0
while (n<10):
    num=int(input("enter a number"))
    GPIO.output(18,True)
    time.sleep(1)
    GPIO.output(18,False)
    time.sleep(1)
    n=n+1

\end{lstlisting}

FInaly we connected two more LEDs to the Pi using GPIO pins 23 and 25. We created a program to blink these LEDs in a patern. The code for this can be seen in listing 2. A circuit digram can be seen in figure 6 and a picture in figure 7.

\begin{figure}[h]
\begin{center}
\includegraphics[width=.4\columnwidth]{lab3-3.png}
\end{center}
\caption{\label{fig:pic} Three LED pi cirucit}
\end{figure}

\begin{figure}[h]
\begin{center}
\includegraphics[width=.4\columnwidth]{pi.jpg}
\end{center}
\caption{\label{fig:pic} Three LED pi picture}
\end{figure}

\begin{lstlisting}[caption=3 LED patern, label=amb]
import RPi.GPIO as GPIO
import time
GPIO.setmode(GPIO.BCM)
GPIO.setup(18, GPIO.OUT)
GPIO.setup(23, GPIO.OUT)
GPIO.setup(25, GPIO.OUT)

GPIO.output(18,True)
time.sleep(1)
GPIO.output(18,False)
time.sleep(1)
GPIO.output(18,True)
time.sleep(1)
GPIO.output(18,False)
GPIO.output(23,True)
time.sleep(1)
GPIO.output(23,False)
time.sleep(1)
GPIO.output(23,True)
time.sleep(1)
GPIO.output(23,False)
GPIO.output(25,True)
time.sleep(1)
GPIO.output(25,False)
time.sleep(1)
GPIO.output(25,True)
time.sleep(1)
GPIO.output(25,False)

\end{lstlisting}


\end{document}