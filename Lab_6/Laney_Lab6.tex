% This is style of APS journal, physicists should aim to use it
\documentclass[prl,12pt,notitlepage,aps,onecolumn,superscriptaddress]{revtex4-1}

%---------------------------------------------------------------
\usepackage{listings} % allow to include nicely formated listings
\usepackage{xcolor}   % we can add and use colors
\usepackage{fullpage} % default article style is to greedy about margins
\usepackage{graphicx} % REQUIRED TO WORK WITH FIGURES
\usepackage{amsmath,amssymb} % much better math and equation handling
\usepackage{booktabs}
\usepackage{float}
%---------------------------------------------------------------

\begin{document}
%---------------------------------------------------------------
\title{Lab 06 Report}
\author{William Laney \& Michael Kopreski}
\date{October 19, 2016}
\maketitle

\section{Group Member Roles}
Circuit Architect: William

Resource Manager (collected and tested components): William

Hardware Specialist (assembled the circuit): William

Programer: Michael

QA Specialist: Michael

\section{Inital set up and testing}
We began this lab by wiring the MCP4725 to the raspbery pi using the I$^2$C bus, a picture can be seen in figure 1. We then wrote code to have the DAC ouput a constan voltage, this code can be seen in the apendex, listing 1. We compared the voltage we asked for from the Pi to the output voltage provided by the DAC. The results of this can be seen in table 1. A plot of the percent error between the requested and actual output can be seen in figure 2. From this figure we see that the error is always below 3\% in the chips operating range, we also see that the chip seems to have lower error between 1.5V to the rail. More data points need to be taken in order to determin if this a true charterstic of the chip.

\begin{figure}[h]
\begin{center}
\includegraphics[width=.3\columnwidth]{cir.jpg}
\end{center}
\caption{\label{fig:pic} Circuit}
\end{figure}

% Table generated by Excel2LaTeX from sheet 'Sheet1'
\begin{table}[htbp]
  \centering
  \caption{DC Data}
    \begin{tabular}{c|c|c|c}
    \toprule
    Input & Expected Output Voltage & Output Voltage & Percent Error \\
    \midrule
    4095  & 4.74884033 & 4.756 & 0.00150767 \\
    3583  & 4.15509033 & 4.176 & 0.0050323 \\
    3071  & 3.56134033 & 3.586 & 0.00692427 \\
    2559  & 2.96759033 & 2.966 & 0.0005359 \\
    2047  & 2.37384033 & 2.386 & 0.00512236 \\
    1535  & 1.78009033 & 1.786 & 0.00331987 \\
    1023  & 1.18634033 & 1.216 & 0.02500098 \\
    767   & 0.88946533 & 0.903 & 0.01521663 \\
    511   & 0.59259033 & 0.588 & 0.00774621 \\
    255   & 0.29571533 & 0.288 & 0.0260904 \\
    0     & 0     & 0.007 & Inf \\
    \bottomrule
    \end{tabular}%
  \label{tab:addlabel}%
\end{table}%


\begin{figure}[h]
\begin{center}
\includegraphics[width=.7\columnwidth]{plot.png}
\end{center}
\caption{\label{fig:pic} Percent Error vs. Expected Output}
\end{figure}

\section{Creating a Function Generator}
Next wrote code that would allow the Pi to output either a square wave or sin wave at verying frequencies. This code can be seen in the apendex in listing 2. We found by looking at the output on the scope that maximum frequncy we could output the square wave was 95$\pm2$Hz. The square wave was stable with just a litte woble at the top of the wave. A picture of thic can be seen in figure ADD NUMBER.

\begin{figure}[h]
\begin{center}
\includegraphics[width=.3\columnwidth]{95sq.jpg}
\end{center}
\caption{\label{fig:pic} 95Hz Square wave from Pi}
\end{figure}

Next we moved on to test the fast fourie tansform (FFT) feature of the scopes. To do this we attached the function generator outputing a 10kHz sin wave to the input of the scope. We then preformed a FFT on this wave form. From it we foudn that the second harmoinc was 40db below the fundenmantal. The largest harmonic was at 20kHz. A picuter of the FFT can be seen in figure ADD NUMBER.

\begin{figure}[h]
\begin{center}
\includegraphics[width=.3\columnwidth]{10kfft.jpg}
\end{center}
\caption{\label{fig:pic} FFT of 10k sine wave form funcion generator}
\end{figure}

Now that we had exprience with the scopes FFT functunality we used this feature on the Pi. We started by finding the highest frequncie sin wave the pi could output. We found this to be about 95Hz. This is constent with the maximum frequncie of the square wave, so this makes sense. We preformed an FFT on this wave form and found the largest harmonic to be 25db. This can be seen in figure ADD NUMBER.

\begin{figure}[h]
\begin{center}
\includegraphics[width=.3\columnwidth]{95fft.jpg}
\end{center}
\caption{\label{fig:pic} FFT of 95Hz sine wave from pi}
\end{figure}

We were not able to increase the maximum frequeny of our sine wave. To do this you would need the Pi to run faster, this could be achomplished with code imporvents, or using a lower level langauge such as C. 

You can decisee the amplitude of the largest harmonic by decreadin the frequncie. A picture of this can be seen in figure ADD NUMBER.

\begin{figure}[h]
\begin{center}
\includegraphics[width=.3\columnwidth]{47fft.jpg}
\end{center}
\caption{\label{fig:pic} FFT of 47Hz sine wave from Pi}
\end{figure}

We were able to generate at 190hz dine wave while keeping the second harmonic at least 20dB lower. However at this frequncie the sine wave was very messy. 

\section{Conclusion}
In this lab we learned about the I$^2$C protical. We also learne how DAC fucntion. Adtionaly we learned about creating wave forms with a DAC and how to use the math functions in the scope to analysises these wave froms. Finaly we learned about the relation between wave from frequicie and the harmonics it produces. 

\section{Appendix}

\lstinputlisting[language=Python, caption=DC voltage, label=amb]{bb_code_lab6.py}

\lstinputlisting[language=Python, caption=Function generator, label=amb]{adc.py}

\end{document}
