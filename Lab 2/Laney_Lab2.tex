% This is style of APS journal, physicists should aim to use it
\documentclass[prl,12pt,notitlepage,aps,onecolumn,superscriptaddress]{revtex4-1}
%---------------------------------------------------------------
\usepackage{listings} % allow to include nicely formated listings
\usepackage{xcolor}   % we can add and use colors
\usepackage{fullpage} % default article style is to greedy about margins
\usepackage{graphicx} % REQUIRED TO WORK WITH FIGURES
\usepackage{amsmath,amssymb} % much better math and equation handling
\usepackage{booktabs}
\usepackage{float}
%---------------------------------------------------------------
%creat command that \ohm creats upcase omega (the ohm symbol)
\newcommand{\ohm}{$\Omega$ }

\begin{document}
%---------------------------------------------------------------
\title{Lab 02 Report}
\author{William Laney \& Jack  Donahue}
\date{September 14, 2016}
\maketitle

\section{Group Member Roles}

Circuit Architect: William Laney

Resource Manager (collected and tested componets):  William Laney

Hardware Specialist (assembled the circuit): Jack  Donahue

QA Specialist: Jack  Donahue

\section{Basic Squaring Function}
We began this lab by creating the simplest squaring function possible with the AD633 mixer. We connected the X2, Y2, and Z inputs to ground and then connected the X1 and Y1 inputs together. This means that ideally the two inputs would be the same so that the would square the input and divide by 10V. A diagram of this circuit can be seen in figure 1. For this chip in order to insure clean voltage rails we attached 0.1$\mu$F capacitor to the voltage inputs to act as a low pass filter. Additionally we connected the input to 7.5k\ohm resistor and 10k\ohm potentiometer to give us a range of input voltages. With this we tested the squaring function over a range of approximately 8V to -8V. We found that the device was more accurate with higher magnitude voltages (above $\pm$1V). Additionally we found that when the input was set to 0 the output was non zero, meaning that there was an offset in the device. All of our results can be seen in table 1, and a picture of the circuit can be seen in figure 2.

% Table generated by Excel2LaTeX from sheet 'original'
\begin{table}[h]
 \centering
 \caption{Squaring function results}
   \begin{tabular}{|c|c|c|c|}
   \toprule
 In (V)   & Out (V)  & Expected (V)  & $|$actual-expected$|$ (V) \\
   \midrule
   8.33  & 6.974 & 0.6974 & 7.6326 \\
   2.01  & 0.4073 & 0.04073 & 1.96927 \\
   1.08  & 0.1202 & 0.01202 & 1.06798 \\
   0.53  & 0.0301 & 0.00301 & 0.52699 \\
   0.24  & 0.0073 & 0.00073 & 0.23927 \\
   0     & 0.0014 & 0.00014 & 0.00014 \\
   -0.27 & 0.0085 & 0.00085 & 0.27085 \\
   -0.49 & 0.0255 & 0.00255 & 0.49255 \\
   -1    & 0.1013 & 0.01013 & 1.01013 \\
   -2.03 & 0.4131 & 0.04131 & 2.07131 \\
   -8.13 & 6.632 & 0.6632 & 8.7932 \\
   \bottomrule
   \end{tabular}%
 \label{tab:addlabel}%
\end{table}%

\begin{figure}[h]
\begin{center}
\includegraphics[width=.4\columnwidth]{square.jpg}
\end{center}
\caption{\label{fig:pic} squaring circuit picture}
\end{figure}

\begin{figure}[h]
\begin{center}
\includegraphics[width=.4\columnwidth]{lab2-1.png}
\end{center}
\caption{\label{fig:pic} squaring circuit diagram}
\end{figure}

\section{Error correction in the chip and offsets}
We found in the last section that our chip did not precisely square our inputs, and that there was an offset visible when we set the inputs to zero. We decided to attempt to minimize this error and to see how prices we could get the output of the chip to be. There are three main sources of error in the chip, offset errors on the input and output, scale factor error, and nonlinearity in the chip. For the purpose of this lab we only focused on the input and output offset. To determine the offset of each input we started by attaching the other input to ground. We then connected the ground associated with the input we were testing to a voltage divider made up of a 1k\ohm resistor and 1k\ohm potentiometer. This allowed us to vary the ground to be near, but not exactly zero values. This offset from zero in ground can correct for the input offset of the chip. We then adjusted this ground voltage until we found a voltage that would allow us to change the input with no effect on the output. Since in this case we are multiplying a voltage by zero we would ideally get zero, but due to the other offsets and errors in the chip we were only looking for a constant output. When we completed this for one input we repeated the procedure for the other input. A circuit diagram for this can be seen in figure 3. Finally we had already found the output offset in the last section, it is the output voltage when both inputs are zero. We found that the X input had an offset of +2.4mV and the Y had an offset of +1mV. The output offset is 1.4mV. We corrected these offsets by using voltage dividers to apply the correct offset voltage in place of ground to the two inputs and the output.

\begin{figure}[h]
\begin{center}
\includegraphics[width=.4\columnwidth]{lab2-2.png}
\end{center}
\caption{\label{fig:pic} Offset test circuit}
\end{figure}

To correct the output however we connected to the Z (the adding input) instead of the actual output. Connecting to the actually output did not properly correct the offset. We then repeated the experiment we did in the last section, varying the voltage from +8V to -8V and recording the output. We found that with the offset correction our values in the $\pm$0-2V range improved in precision we lost precision at the $\pm$8V range. This means that our corrections are not useful at the higher voltages. A full table of our results can be seen in table 2. A circuit diagram including our offset correction can be seen in figure 4.

% Table generated by Excel2LaTeX from sheet 'corrected'
\begin{table}[h]
 \centering
 \caption{Squaring function with offset correction results}
   \begin{tabular}{|c|c|c|c|}
   \toprule
 In (V)   & Out (V)  & Expected (V)  & $|$actual-expected$|$ (V) \\
   \midrule
   8.306 & 6.85  & 0.685 & 7.621 \\
   1.993 & 0.397 & 0.0397 & 1.9533 \\
   1.002 & 0.1   & 0.01  & 0.992 \\
   0.4967 & 0.0243 & 0.00243 & 0.49427 \\
   0.2505 & 5.90E-03 & 0.00059 & 0.24991 \\
   0     & 5.00E-04 & 0.00005 & 0.00005 \\
   -0.245 & 5.50E-03 & 0.00055 & 0.24555 \\
   -0.501 & 2.42E-02 & 0.00242 & 0.50342 \\
   -1.032 & 1.05E-01 & 0.01053 & 1.04253 \\
   -2.021 & 4.05E-01 & 0.0405 & 2.0615 \\
   -8.116 & 6.53E+00 & 0.653 & 8.769 \\
   \bottomrule
   \end{tabular}%
 \label{tab:addlabel}%
\end{table}%

\begin{figure}[h]
\begin{center}
\includegraphics[width=.4\columnwidth]{lab2-3.png}
\end{center}
\caption{\label{fig:pic} Offset corrected circuit}
\end{figure}

\section{square root function}
In this section of the lab we attempted to create a square root function using our squaring function and a feedback loop with an LM741 op-amp. To do this we connected our input voltage to the inverting input of the op-amp through a 10k\ohm resistor. The non inverting output was connected to ground. We then connected output of the op-amp to the input of our squaring function through a diode. Additionally we connected another 10k\ohm from the inverting input of the op-amp to the output of the squaring function to create the feedback loop. This created a function the took the input voltage, multiplied it by ten, and then took the square root. To get this output we measured from the input to the squaring function. In addition to the circuit laid out above we left in the offset correction century in this circuit. A picture of the circuit can be seen in figure 5, and a circuit diagram in figure 6. We tested this circuit in the range of 0 to -6V, and the results of this can be seen in table 3. From these results we found that the circuit performs well in this range, so this circuit does not have a failure case over the voltages needed in this lab.

% Table generated by Excel2LaTeX from sheet 'sqrt'
\begin{table}[h]
 \centering
 \caption{Square root function results}
   \begin{tabular}{|c|c|c|c|}
   \toprule
   In (V)   & Out (V)  & Expected (V)  & $|$actual-expected$|$ (V) \\
   \midrule
   0.0004 & 0.2   & 0.063246 & 0.136754 \\
   0.1001 & 1     & 1.0005 & 0.0005 \\
   0.3968 & 1.98  & 1.991984 & 0.011984 \\
   0.9007 & 2.98  & 3.001166 & 0.021166 \\
   2.537 & 5     & 5.036864 & 0.036864 \\
   4.909 & 6.95  & 7.006426 & 0.056426 \\
   5.765 & 7.54  & 7.59276 & 0.05276 \\
   \bottomrule
   \end{tabular}%
 \label{tab:addlabel}%
\end{table}%

\begin{figure}[h]
\begin{center}
\includegraphics[width=.4\columnwidth]{square_root.jpg}
\end{center}
\caption{\label{fig:pic} square root circuit picture}
\end{figure}

\begin{figure}[h]
\begin{center}
\includegraphics[width=.4\columnwidth]{lab2-4.png}
\end{center}
\caption{\label{fig:pic} square root circuit diagram}
\end{figure}

\section{Conclusion}
In this lab we learned about the function of mixers and how to use them. Additionally we learned how to look for and correct offsets in a chip. Finally we learned how to use op-amps and feedback loops to invert functions.
\end{document}

