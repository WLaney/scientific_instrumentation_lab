% This is style of APS journal, physicists should aim to use it
\documentclass[prl,12pt,notitlepage,aps,onecolumn,superscriptaddress]{revtex4-1}

%---------------------------------------------------------------
\usepackage{listings} % allow to include nicely formated listings
\usepackage{xcolor}   % we can add and use colors
\usepackage{fullpage} % default article style is to greedy about margins
\usepackage{graphicx} % REQUIRED TO WORK WITH FIGURES
\usepackage{amsmath,amssymb} % much better math and equation handling
\usepackage{booktabs}
\usepackage{float}
%---------------------------------------------------------------

\begin{document}
%---------------------------------------------------------------
\title{Lab 01 Report}
\author{William Laney \& Ben Kincaid}
\date{September 7, 2016}
\maketitle

\section{Group Member Roles}

Circuit Architect: Ben Kincaid

Resource Manager (collected and tested componets): Ben Kincaid

Hardware Specialist (assembled the circuit): William Laney

QA Specialist William Laney


\section{Simple LED}
We began by creating a circuit that uses a LM741 op-amp to drive a Green LED. When the LED turned on we got a reading of 1.8V across it. A picture of this circuit and a circuit diagram specifying the value of components used can be seen below.

\begin{figure}[H]
\begin{center}
\includegraphics[width=.4\columnwidth]{LED.jpg}
\end{center}
\caption{\label{fig:pic} Simple op-amp LED circuit picture}
\end{figure}

\begin{figure}[H]
\begin{center}
\includegraphics[width=.4\columnwidth]{circuit1.png}
\end{center}
\caption{\label{fig:pic} Simple op-amp LED circuit digrame. The red line is current.}
\end{figure}

\section{Photodiode Circuit}
Next we added the photodiode circuit directly next to our LED circuit. We changed our LED from green to red in order to get a better reading on the photodiode. We found that with no LED we read $\approx 312$mV and with the LED on $\approx 350$mV. A picture of the photodiode circuit next to the LED circuit can be seen below, along with a circuit diagram specifying values of components used.

\begin{figure}[H]
\begin{center}
\includegraphics[width=.4\columnwidth]{led+photodiode.jpg}
\end{center}
\caption{\label{fig:pic} Simple op-amp LED circuit with photodiode picture}
\end{figure}

\begin{figure}[H]
\begin{center}
\includegraphics[width=.4\columnwidth]{circuit2.png}
\end{center}
\caption{\label{fig:pic} Simple op-amp LED circuit with photodiode diagram The red line is current.}
\end{figure}

\section{Potentiometer Circuit}
Next we wired the 10k potentiometer into the inverting input of the LED in order to control the brightness of the LED. A picture of this and a circuit diagram with values can be seen below.

\begin{figure}[H]
\begin{center}
\includegraphics[width=.4\columnwidth]{pot.jpg}
\end{center}
\caption{\label{fig:pic} Potentiometer LED brightness control picture}
\end{figure}

\begin{figure}[H]
\begin{center}
\includegraphics[width=.4\columnwidth]{circuit3.png}
\end{center}
\caption{\label{fig:pic} Potentiometer LED brightness control diagram. The red line is current.}
\end{figure}

\section{Feedback circuit with potentiometer}
For the next section of the lab we used a second LM741 op-amp wired into the photodiode circuit to create a feedback circuit for the LED. The photentiometer was also wired into this second op-amp. This allowed us to control the brightness of the LED. However before we were controlling the brightness directly now we controlled it by amplifying the output of the photodiode by different amounts. The brighter the LED was the higher the voltage read off of the photodiode and this output can be used to control the LED.

A picture of this circuit along with a circuit diagram can be seen below.

\begin{figure}[H]
\begin{center}
\includegraphics[width=.4\columnwidth]{pot_feedback.jpg}
\end{center}
\caption{\label{fig:pic} LED brightness feedback picture}
\end{figure}

\begin{figure}[H]
\begin{center}
\includegraphics[width=.4\columnwidth]{circuit4.png}
\end{center}
\caption{\label{fig:pic} Potentiometer LED brightness feedback diagram. The red line is current, keep in mind that no current enters the op-amp but current can be outputted.}
\end{figure}

\section{Feedback with unstable power supply}
In this section we replaced the ground connection at the non-inverting input of the LED driving op-amp with the output of a function generator. This creates wobble in the brightness of the LED, however the feedback circuit from the previous section should correct this wobble and make the LED appear as a constant light source. Our circuit was able to correct for the wobble introduced by the function generator at some amplitudes of frequencies outputted by the function generator, but not all. A table showing the failure points of our feedback circuit can be seen below. Additionally a picture of the circuit and a circuit diagram with values are below.

% Table generated by Excel2LaTeX from sheet 'Sheet1'
\begin{table}[h]
 \centering
 \caption{Feedback Reliability, frequency and voltage fed inplace of ground}
   \begin{tabular}{|l|l|}
   \toprule
   1Hz Range & $\approx 400$mV -17V on steady, 17$<$V$<$20 blinking \\
   \midrule
   10Hz Range & $\approx 400$mV -17V on steady, 17$<$V$<$20 blinking \\
   100 Hz Range & $\approx 400$mV -15V on steady, 15$<$V$<$20 blinking clearly on scope, but can not be seen visually \\
   1k Hz Range & $\approx 400$mV -15.8V on steady, 15$<$V$<$20 blinking clearly on scope, but can not be seen visually \\
   10k Hz Range & $\approx 400$mV -18V on steady, 18$<$V$<$20 blinking very quickly, can be seen visually \\
   100k Hz Range & $\approx 400$mV -20V on steady \\
   \bottomrule
   \end{tabular}%
 \label{tab:addlabel}%
\end{table}%

\begin{figure}[H]
\begin{center}
\includegraphics[width=.4\columnwidth]{function_gen_pot.jpg}
\end{center}
\caption{\label{fig:pic} LED brightness feedback with messy voltage picture}
\end{figure}

\begin{figure}[H]
\begin{center}
\includegraphics[width=.4\columnwidth]{circuit5.png}
\end{center}
\caption{\label{fig:pic} Potentiometer LED brightness feedback with messy voltage diagram. The red line is current, keep in mind that no current enters the op-amp but current can be outputted.}
\end{figure}

\section{Conclusion}
In this lab we learned how to use op-amps to use feedback and create a comparator and correct for small fluctuations from a source. This can be useful to help clean up bad signals, and to amplify signals.
\end{document}
