% This is style of APS journal, physicists should aim to use it
\documentclass[prl,12pt,notitlepage,aps,onecolumn,superscriptaddress]{revtex4-1}


%---------------------------------------------------------------
\usepackage{listings} % allow to include nicely formated listings
\usepackage{xcolor}   % we can add and use colors
\usepackage{fullpage} % default article style is to greedy about margins
\usepackage{graphicx} % REQUIRED TO WORK WITH FIGURES
\usepackage{amsmath,amssymb} % much better math and equation handling
\usepackage{booktabs}
\usepackage{float}
%---------------------------------------------------------------
\newcommand{\ohm}{$\Omega$ }


\begin{document}
%---------------------------------------------------------------
\title{Lab 07 Report}
\author{William Laney \& Ian Huber}
\date{November 2, 2016}
\maketitle


\section{Group Member Roles}
Resource Manager (collected and tested components): Ian


Hardware Specialist (assembled the circuit): Ian


Programer: William


QA Specialist William


\section{Initial setup and testing}
We began this lab by simply trying to get the mbed to communicate via serial with a computer. We did this using the terminal utility screen. We first wrote code that simply printed data to the serial monitor. We then made code that would echo a serial string input from the computer. Final we wrote code that would take an integer and a floating point number, add a fixed value to them, and output the numbers back to serial. The code for this can be seen in listing 1.


\section{analog voltage}
Next we wrote code for the mbed that would look at an analogue voltage input and write the voltage to the serial monitor. This code can be seen in listing 2. We tested this functionality from 0 to 3.3V and compared the mbed readings to a digital multimeter. A plot showing the percent error for each of these measurements can be seen in figure 1. From this we concluded that the error in the mbed is small and random. We were not able to find systematic error.


\begin{figure}[h]
\begin{center}
\includegraphics[width=1\columnwidth]{vin.png}
\end{center}
\caption{\label{fig:pic} Input voltage vs. Percent error}
\end{figure}


We then did the same experiment in reverse. That is to say we wrote a code that would take a user input through serial and output that voltage. This code can be seen in listing 3. We compared the requested output voltage to the output voltage read by the mbed. A plot showing the percent error for each of these measurements can be seen in figure 2. From this we concluded that the mbed gains accuracy exponentially between 0 and 1 volt and any error over 1 volt is random.


\begin{figure}[h]
\begin{center}
\includegraphics[width=1\columnwidth]{vout.png}
\end{center}
\caption{\label{fig:pic} Requested output voltage vs. Percent error}
\end{figure}


\section{PWM and interrupts}
We began this section by testing to see what was the fastest square wave we could generate. This code can be seen in listing 4. We found that we could generate a square wave up to 960 Ghz.


Next we used code written by Dr. Cooke that generates an output pulse that triggers an interrupt and flips a digital output pin. This code can be found in listing 5. A picture of the output of this code can be seen in figure 3. From this we found that the delay between the triggering edge and the resulting action taken by the interrupt service routine (ISR) is 6.7$\pm0.2\mu$s. This value is inherent to the system and can not be overcome using the techniques in this lab. There are ways to shorten this time laid out in the mbed user manual, but they are outside the scope of this class.


\begin{figure}[h]
\begin{center}
\includegraphics[width=.5\columnwidth]{flip.jpg}
\end{center}
\caption{\label{fig:pic} ISR causing a digital output to flip}
\end{figure}


Finally we wrote code that triggers on the edge of a PWM output and generates two pulses. We found that the shortest time between pulse we could achieve was 160ns. This time stayed constant throughout repeated trials however the output pulses are not square. This can be seen in figure 4. At higher time separations between the pulses they form better square waves. The code we used to generate these pulses can be seen in listing 5. In this lab we generated the two pulses on the same line, however it is simple to create the start and stop pulses on different lines and this may be more useful in a laboratory setting for controlling equipment.


\begin{figure}[h]
\begin{center}
\includegraphics[width=.5\columnwidth]{pulse.jpg}
\end{center}
\caption{\label{fig:pic} On, off pulses}
\end{figure}


\section{Rotary encoder}
The final section of this lab was to create code that would increment a count based on a rotary encoder. I had already done this for the Pi as a prelab for Lab 5, however we found that the pi was not fast enough to accurately read the rotary encoder. In discovering this I did create functional code for reading the rotary encoder in python, which can be seen in listing 6. A circuit diagram for the encoder can be seen in figure 5.


\begin{figure}[h]
\begin{center}
\includegraphics[width=.6\columnwidth]{rotary.png}
\end{center}
\caption{\label{fig:pic} rotary encoder schematic}
\end{figure}


For this lab I began by translating my python code into C++ for the mbed. However when we tried to run this code connected to the mbed we found that it did not function properly. Looking at the waveforms from the rotary encoder on the scope we found that the rotary encoder was becoming stuck on the on state and giving very messy output in general. We tried to clean up this output at first by putting 0.1uF capacitors to ground. This helped a little but not enough to get reliable reading from the rotary encoder. We then built an RC filter using 4.7k\ohm along with the 0.1uF capacitor. This did improve the signal but it still was not good enough. Finally we switched out our rotary encoder for a new one, leaving the filters, and this ended up solving our problem.


After this we release our code was also not functioning properly. The interrupts were happening too fast and were causing both the first and second falling edges from the two sides of the encoder to be counted, this caused every turn of the encoder to raise and then lower the value by one. To solve this we moved where our wait statements were in the code. Our final working code can be seen in listing 7.


\section{Appendix}


\lstinputlisting[language=c++, caption=Serial testing, label=amb]{string_reverse.cpp}


\lstinputlisting[language=c++, caption=Read analog voltage, label=amb]{read_vin.cpp}


\lstinputlisting[language=c++, caption=Write analog voltage, label=amb]{AnalogOut_DAC_Test.cpp}


\lstinputlisting[language=c++, caption=PWM out, label=amb]
{PMW_out.cpp}


\lstinputlisting[language=c++, caption=ISR, label=amb]
{ISR_Example.cpp}


\lstinputlisting[language=Python, caption=python rotarty encoder, label=amb]{laney5_rotary.py}


\lstinputlisting[language=c++, caption=C++ rotary encoder, label=amb]
{rotary.cpp}


\end{document}